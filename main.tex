\documentclass[
article,			% Artigo científico
12pt,				% Tamanho da fonte
oneside,			% Impressão apenas em um lado
a4paper,			% Tamanho do papel
english,			% Idioma adicional
brazil				% Idioma principal
]{abntex2}


\usepackage{cmap}				% Caracteres especiais
\usepackage{lmodern}			% Usa a fonte Latin Modern
\usepackage[T1]{fontenc}		% Selecao de codigos de fonte
\usepackage[utf8]{inputenc}		% Codificacao do documento
\usepackage{indentfirst}		% Indenta o primeiro parágrafo
\usepackage{nomencl} 			% Lista de simbolos
\usepackage{color}				% Controle das cores
\usepackage{graphicx}			% Inclusão de gráficos
\usepackage{microtype} 			% para melhorias de justificação
\usepackage{float}              % Para posicionar imagens [H]
\usepackage{booktabs}           % Tabelas bonitas (toprule, midrule)
\usepackage{amsmath}            % Fórmulas matemáticas


\begin{document}
	
	\begin{center}
		\textbf{\large Entre o Porto e a Cidade: O Desacoplamento da Demanda na Travessia Santos-Guarujá (2019-2025)}
		
		\vspace{0.5cm}
		
		\textit{\large Between the Port and the City: Demand Decoupling in the Santos-Guarujá Crossing (2019-2025)}
		
		\vspace{1cm}
		
		\textbf{Fernando Gomes Cruz} \\
		\small{Faculdade de Tecnologia do Estado de São Paulo, Rubens Lara} \\
		\small{fernando.cruz7@fatec.sp.gov.br} \\
		\small{Santos, 2025}
		\vspace{1cm}
	\end{center}
	
	\begin{resumoumacoluna}
		\begin{otherlanguage*}{portuguese}
			A travessia de balsas Santos-Guarujá representa um dos principais gargalos logísticos do litoral paulista. Este trabalho propõe uma abordagem de Ciência de Dados para investigar os determinantes do fluxo de automóveis nesta travessia entre 2019 e 2025. O objetivo principal foi testar a hipótese de que a atividade econômica portuária (importações) atua como preditora da demanda de mobilidade local. A metodologia envolveu a integração de bases de dados heterogêneas, modelagem de sazonalidade discreta por efeitos fixos mensais e a aplicação comparativa de modelos de \textit{Gradient Boosting} (XGBoost) e Regressão Estrutural Bayesiana. O modelo final obteve um $R^2$ de 0,33 e MAPE de 9,29\%. Os resultados indicaram que o choque da pandemia de COVID-19 foi um dos fatores determinantes na variância dos dados, gerando uma retração estrutural no sistema, apresentando a pior tendência secular entre as principais travessias do estado. Contrariando a hipótese inicial, a correlação com a atividade econômica mostrou-se nula, confirmando um desacoplamento entre o crescimento do porto e a mobilidade urbana da balsa, indicativo de saturação da infraestrutura.
			
			\vspace{\onelineskip}
			
			\noindent
			\textbf{Palavras-chave}: Ciência de Dados. Mobilidade Urbana. Séries Temporais. Inferência Bayesiana. Travessia Santos-Guarujá.
		\end{otherlanguage*}
	\end{resumoumacoluna}
	
	\begin{resumoumacoluna}[Abstract]
		\begin{otherlanguage*}{english}
			The Santos-Guarujá ferry crossing represents one of the main logistical bottlenecks on the coast of São Paulo. This work proposes a Data Science approach to investigate the determinants of automobile flow in this crossing between 2019 and 2025. The main objective was to test the hypothesis that port economic activity (imports) acts as a predictor of local mobility demand. The methodology involved the integration of heterogeneous databases, discrete seasonality modeling via monthly fixed effects, and the comparative application of Gradient Boosting (XGBoost) and Bayesian Structural Regression models. The final model achieved an $R^2$ of 0.33 and MAPE of 9.29\%. Results indicated that the COVID-19 pandemic shock was one of the determining factors in data variance, generating a structural retraction in the system, presenting the worst secular trend among the state's main crossings. Contrary to the initial hypothesis, the correlation with economic activity proved null, confirming a decoupling between port growth and ferry urban mobility, indicative of infrastructure saturation.
			
			\vspace{\onelineskip}
			
			\noindent
			\textbf{Keywords}: Data Science. Urban Mobility. Time Series. Bayesian Inference. Santos-Guarujá Crossing.
		\end{otherlanguage*}
	\end{resumoumacoluna}
	
	\section{Introdução}
	
	
	A interface entre grandes complexos portuários e os tecidos urbanos que os acolhem constitui, contemporaneamente, um dos desafios mais intrincados da geografia urbana e do planejamento regional. No litoral do estado de São Paulo, a Baixada Santista epitomiza essa tensão dialética com uma magnitude ímpar no hemisfério sul: abriga o Porto de Santos, o maior da América Latina, responsável por movimentar cerca de um terço da balança comercial brasileira \cite{antagonismo}. Contudo, esta infraestrutura de escala continental coexiste, em um espaço geograficamente exíguo e fragmentado por estuários, com uma densa conurbação urbana que demanda mobilidade e eficiência nos deslocamentos diários.
	
	O epicentro físico e simbólico desse conflito reside na travessia marítima entre os municípios de Santos e Guarujá, um gargalo logístico histórico onde as dinâmicas globais de exportação colidem frontalmente com as necessidades locais de pendularidade da força de trabalho. Realizada predominantemente por um sistema de balsas (\textit{ferry-boats}) operado pelo Departamento Hidroviário (DH), a travessia opera frequentemente no limite de sua saturação operacional. Sujeito a interrupções por condições meteorológicas e manutenção de embarcações, o sistema impõe custos sociais severos, materializados em filas de espera que penalizam a produtividade regional \cite{tribuna_caos}.
	
	\subsection{O Choque Exógeno da Pandemia}
	
	O advento da pandemia de COVID-19, no início de 2020, introduziu um vetor de perturbação exógena sem precedentes neste sistema. A crise sanitária operou como um "experimento natural", impondo pressões opostas sobre as funções da região. Por um lado, decretos de quarentena restringiram a circulação urbana \cite{scielo_mobilidade}; por outro, a atividade portuária foi classificada como essencial, mantendo sua operação para garantir as cadeias de suprimentos \cite{unisanta}.
	
	Este fenômeno sugere uma dicotomia observável: a mobilidade urbana (elástica e sensível ao risco de contágio) pode ter se desvinculado da dinâmica portuária (inelástica e baseada na demanda externa de commodities). A hipótese central deste trabalho é a de um "desacoplamento": investigar se, durante e após a crise (período 2019-2025), a correlação histórica entre o crescimento do porto e o fluxo da balsa foi rompida.
	
	\subsection{Abordagem por Ciência de Dados}
	
	Investigar essa dinâmica exige superar ferramentas analíticas tradicionais. Modelos clássicos de previsão, muitas vezes fundamentados em pressupostos de linearidade, mostram-se insuficientes para capturar rupturas estruturais abruptas como as provocadas pela pandemia \cite{stitchfix}. A volatilidade dos dados e a introdução de variáveis exógenas exigem uma abordagem baseada em Ciência de Dados.
	
	Nesse sentido, o presente estudo propõe a aplicação de duas técnicas complementares sobre dados públicos do período de 2019 a 2025:
	\begin{enumerate}
		\item \textbf{Gradient Boosting (XGBoost):} Algoritmo de aprendizado de máquina utilizado para mensurar a importância relativa das variáveis (econômicas vs. temporais) na explicação do fluxo.
		\item \textbf{Modelagem Estrutural Bayesiana (BSTS via PyMC):} Uma abordagem probabilística que permite decompor a série temporal em tendência, sazonalidade e regressores externos, quantificando a incerteza e isolando estatisticamente o impacto causal da pandemia e da economia \cite{google_bsts}.
	\end{enumerate}
	
	O objetivo deste trabalho é triplo: (1) construir um \textit{pipeline} de dados que integre fontes de transporte e economia; (2) quantificar o "custo de mobilidade" da pandemia em desvios-padrão; e (3) testar a hipótese do desacoplamento porto-cidade, verificando se as variáveis econômicas perderam poder explicativo sobre a mobilidade urbana recente.
	
	
	\section{Referencial Teórico}
	
	A fundamentação teórica desta investigação é construída sobre a interseção de três domínios do conhecimento: a geografia portuária, com foco na relação porto-cidade; a economia dos transportes sob o impacto de choques exógenos; e a ciência de dados aplicada a séries temporais, abrangendo métodos de aprendizado de máquina e inferência bayesiana.
	
	\subsection{A Dialética Porto-Cidade: Desacoplamento e Conflito}
	
	A relação histórica entre cidades e portos é descrita na literatura como uma trajetória evolutiva que oscila entre a simbiose inicial e o conflito contemporâneo. No modelo clássico "Anyport", a gênese das cidades portuárias é marcada por uma forte integração espacial: o porto é o centro da vida urbana \cite{bird_anyport}. No entanto, a evolução tecnológica do transporte marítimo — notadamente a conteneirização — impulsionou a migração das instalações para áreas distantes dos centros, em busca de grandes espaços livres \cite{hoylet}.
	
	Este fenômeno, denominado "desacoplamento" (\textit{decoupling}), cria uma tensão paradoxal. Enquanto o porto se globaliza, atendendo a uma demanda exógena (comércio internacional), a cidade permanece enraizada em dinâmicas locais. Em Santos, essa separação é exacerbada pela geografia estuarina. A população identifica o porto como fonte de riqueza, mas também como gerador de externalidades negativas, como congestionamentos que competem com o transporte público \cite{pedrosa_santos}.
	
	A hipótese de desacoplamento sugere que, em momentos de crise, as trajetórias de demanda do porto e da cidade podem divergir. Enquanto a cidade para devido a uma crise sanitária, o porto, impulsionado pela demanda inelástica de \textit{commodities}, continua a operar, rompendo as correlações históricas de tráfego \cite{conflito_espaco}.
	
	\subsection{Mobilidade em Gargalos Logísticos}
	
	A mobilidade em regiões estuarinas apresenta desafios singulares. A Travessia Santos-Guarujá opera como um "monopólio natural" da mobilidade local, sem alternativas viárias competitivas de curta distância. A literatura técnica classifica o sistema como saturado: o Volume Médio Diário (VMD) de veículos coloca o sistema sob estresse constante.
	
	A dependência de um modal sujeito a variáveis ambientais (marés) e mecânicas introduz uma incerteza que penaliza a produtividade regional. Até a concretização de soluções estruturais, como o túnel imerso, a análise deve focar na resiliência do sistema atual de balsas frente a choques de demanda \cite{tribuna_caos}.
	
	\subsection{O Impacto da COVID-19 na Demanda}
	
	A pandemia de COVID-19 provocou o maior choque negativo na demanda de transporte de passageiros da história moderna. Estudos indicam reduções de \textit{ridership} no transporte público entre 70\% e 90\% durante os primeiros \textit{lockdowns} \cite{scielo_covid}.
	
	Contudo, esse impacto foi assimetricamente distribuído. Enquanto a mobilidade discricionária (lazer) despencou, a mobilidade de cargas mostrou-se resiliente. A literatura aponta para uma recuperação pós-pandemia em formato "K", onde o uso de transporte individual recupera-se mais rapidamente do que o transporte coletivo ou compartilhado (como balsas de pedestres), devido à persistência de novos hábitos, como o trabalho remoto \cite{jornal_usp}.
	
	\subsection{Modelagem de Séries Temporais: Do Clássico ao Bayesiano}
	
	A análise de fenômenos complexos como a mobilidade pandêmica exige ferramentas analíticas robustas. Os métodos econométricos tradicionais (como ARIMA) pressupõem linearidade e muitas vezes falham em capturar rupturas estruturais abruptas \cite{stitchfix_arima}. A literatura recente aponta para a superioridade de abordagens híbridas.
	
	\subsubsection{Gradient Boosting e Importância de Variáveis}
	Algoritmos de \textit{Gradient Boosting} emergiram como referência para problemas de regressão em dados tabulares. Estes métodos constroem sequencialmente árvores de decisão, onde cada nova árvore visa corrigir os erros residuais das anteriores.
	
	A principal vantagem desta abordagem para este estudo não é apenas a predição, mas a interpretabilidade via \textit{Feature Importance} (Importância de Atributos). Métricas baseadas em ganho de informação permitem ranquear quais fatores — a tendência temporal, a economia ou a pandemia — são os determinantes primários da variância do tráfego, testando empiricamente quais variáveis "importam" mais para o sistema \cite{xgboost_paper}.
	
	\subsubsection{Séries Temporais Estruturais Bayesianas (BSTS)}
	Enquanto o Machine Learning foca na predição pontual, a abordagem Bayesiana foca na quantificação da incerteza e na decomposição estrutural. O modelo BSTS (\textit{Bayesian Structural Time Series}) trata a série temporal observada como uma soma de componentes latentes (tendência, sazonalidade, regressão) que evoluem no tempo \cite{google_bsts}.
	
	A equação estrutural básica pode ser definida como:
	\begin{equation}
		Y_t = \mu_t + S_t + \beta X_t + \epsilon_t
	\end{equation}
	Onde $\mu_t$ é a tendência, $S_t$ a sazonalidade e $X_t$ os regressores externos. Utilizando amostragem MCMC (\textit{Markov Chain Monte Carlo}), é possível estimar não apenas o valor médio desses componentes, mas a distribuição de probabilidade inteira (Intervalos de Credibilidade). Isso é fundamental para afirmar, com rigor estatístico, se uma queda de demanda é uma mudança estrutural significativa ou apenas ruído aleatório do sistema \cite{gelman_bayes}.
	
	\section{Metodologia}
	
	A estratégia metodológica deste estudo foi desenhada para extrair inteligência analítica de fontes de dados governamentais dispersas, integrando-as em um \textit{pipeline} de Ciência de Dados reprodutível. O estudo foi conduzido utilizando a linguagem Python, empregando bibliotecas de manipulação de dados (\textit{Pandas}), aprendizado de máquina (\textit{Scikit-Learn}) e programação probabilística (\textit{PyMC}).
	
	A abordagem é quantitativa e estruturada em três fases principais: Aquisição e Tratamento de Dados (ETL), Engenharia de Atributos e Modelagem Híbrida.
	
	\subsection{Aquisição e Tratamento de Dados (ETL)}
	A construção do \textit{dataset} analítico exigiu a federação de dados de domínios distintos — mobilidade e economia — abrangendo o período de janeiro de 2019 a outubro de 2025.
	
	\begin{itemize}
		\item \textbf{Dados de Mobilidade:} A fonte primária foi o Portal de Dados Abertos do Estado de São Paulo, mantido pela Secretaria de Meio Ambiente, Infraestrutura e Logística (SEMIL). Para isolar a dinâmica turística e pendular motorizada de ruídos causados pela mobilidade ativa, o estudo restringiu o escopo de análise exclusivamente à categoria Automóveis, descartando pedestres e ciclistas.
		
		\item \textbf{Dados Econômicos:} Para a atividade portuária, utilizou-se a série de importações (Valor FOB) para os municípios de Santos e Guarujá. Os dados, originários do Sistema Integrado de Comércio Exterior (Siscomex/MDIC), foram obtidos através do repositório da Fundação Seade e agregados mensalmente.
	\end{itemize}
	
	O processo de ETL envolveu a limpeza de \textit{outliers} operacionais e a sincronização temporal das bases. Para contextualizar o desempenho de Santos-Guarujá, foram também processados dados das demais travessias litorâneas do estado (como São Sebastião/Ilhabela e Cananéia) para fins de comparação de tendências seculares.
	
	\subsection{Engenharia de Atributos e Sazonalidade}
	A modelagem da sazonalidade exigiu uma abordagem robusta para capturar o comportamento assimétrico do turismo litorâneo, onde picos de demanda (como dezembro e janeiro) não seguem necessariamente uma curva suave.
	
	Diferentemente de abordagens clássicas que utilizam funções trigonométricas (séries de Fourier), optou-se por uma modelagem de Sazonalidade Discreta via Efeitos Fixos Mensais. Foi criado um vetor de índices $m \in \{0, ..., 11\}$ representando os meses do ano, permitindo que o modelo probabilístico estimasse um coeficiente independente para cada mês. Essa técnica permite capturar a magnitude exata dos picos de veraneio sem impor a rigidez de uma onda senoidal.
	
	Adicionalmente, as variáveis contínuas (volume de tráfego e valor FOB) foram padronizadas (\textit{z-score}) para facilitar a convergência dos algoritmos e a comparação de coeficientes.
	
	\subsection{Estratégia Analítica: Modelagem Híbrida}
	A análise foi conduzida em três etapas complementares:
	
	\subsubsection{Análise Comparativa de Tendências}
	Preliminarmente, aplicou-se uma regressão linear simplificada a todas as travessias do estado para extrair o componente de tendência secular ($\beta_{trend}$) de cada sistema. O objetivo foi criar um \textit{ranking} de crescimento, permitindo verificar se a estagnação do fluxo é um fenômeno generalizado do litoral ou específico do sistema Santos-Guarujá.
	
	\subsubsection{Seleção de Variáveis com Gradient Boosting}
	Utilizou-se o algoritmo \textit{XGBoost Regressor} para mensurar a importância relativa das variáveis preditoras. Esta etapa de aprendizado de máquina serviu para validar a hierarquia de influência dos atributos, testando empiricamente se a variável econômica (Valor FOB) oferecia ganho de informação superior às variáveis temporais e pandêmicas.
	
	\subsubsection{Inferência Estrutural Bayesiana}
	Para a modelagem explicativa final, adotou-se a abordagem de Séries Temporais Estruturais Bayesianas (BSTS) utilizando a biblioteca \textit{PyMC}. A inferência Bayesiana permitiu incorporar incertezas nos parâmetros e definir a estrutura do fluxo ($Y_t$) através da seguinte equação de regressão:
	
	\begin{equation}
		\mu_t = \alpha + \beta_{trend} \cdot t + \beta_{econ} \cdot X_{imp} + \beta_{pan} \cdot D_{pan} + \beta_{saz}[m_t]
	\end{equation}
	
	Onde:
	\begin{itemize}
		\item $\beta_{trend}$: coeficiente da tendência secular linear;
		\item $\beta_{econ}$: elasticidade do fluxo em relação às importações;
		\item $\beta_{pan}$: impacto do choque exógeno da pandemia (variável \textit{dummy});
		\item $\beta_{saz}[m_t]$: vetor de 12 coeficientes latentes, onde $m_t$ é o índice do mês no instante $t$.
	\end{itemize}
	
	Para a função de verossimilhança (\textit{Likelihood}), manteve-se a distribuição \textbf{Student-t} para conferir robustez contra \textit{outliers}, comuns em dados de transporte sujeitos a interrupções operacionais. A inferência foi realizada via amostragem MCMC (\textit{Markov Chain Monte Carlo}) com 2.000 amostras, gerando Intervalos de Credibilidade de Alta Densidade (HDI) para cada componente estrutural.
	
\section{Resultados e Discussão}

A aplicação da metodologia proposta permitiu caracterizar o perfil da travessia Santos-Guarujá e quantificar os impactos estruturais ocorridos entre janeiro de 2019 e outubro de 2025. É fundamental ressaltar que a análise restringiu-se à categoria de Automóveis, isolando ruídos de outras categorias (como caminhões ou ciclistas) para capturar com maior fidelidade a dinâmica da mobilidade pendular e turística.

\subsection{Tipologia e Contexto Sistêmico}

A análise exploratória inicial buscou situar a Travessia Santos-Guarujá dentro do ecossistema hidroviário estadual. Ao correlacionar a magnitude do fluxo (escala logarítmica) com o Coeficiente de Variação (CV), observou-se uma distinção clara entre travessias de perfil turístico e travessias de perfil urbano.

Conforme ilustrado na Figura \ref{fig:tipologia}, o sistema Santos-Guarujá destaca-se como um \textit{outlier}. O sistema apresenta o maior volume absoluto de tráfego do estado, mas, paradoxalmente, possui um dos menores índices relativos de sazonalidade ($CV < 0,20$). Diferentemente de travessias como Ilhabela ou Cananéia, sujeitas a picos extremos de veraneio e vales profundos, a travessia do Porto opera com uma demanda basal elevada e inelástica durante todo o ano, característica típica de deslocamentos urbanos obrigatórios.

\begin{figure}[H]
	\centering
	\caption{Tipologia das Travessias: Volume vs. Sazonalidade (Automóveis)}
	\includegraphics[width=0.9\linewidth]{assets/fig1_tipologia.png}
	\fonte{Elaborado pelo autor.}
	\label{fig:tipologia}
\end{figure}

Esta constatação sugere que o sistema opera próximo ao limite de sua capacidade física. Nesse cenário, aumentos na demanda potencial tendem a se converter em tempo de fila, e não necessariamente em volume transportado, devido à rigidez da oferta de balsas.

\subsection{Análise Comparativa de Tendências Seculares}

Para verificar a saúde operacional do sistema, isolou-se a tendência secular ($\beta_{trend}$) de todas as travessias litorâneas monitoradas, descontando-se os efeitos sazonais e o choque pontual da pandemia.

Os resultados, apresentados na Figura \ref{fig:ranking}, revelaram uma divergência geográfica significativa. Enquanto travessias situadas nas extremidades do litoral (como São Sebastião/Ilhabela e Iguape) apresentaram coeficientes positivos — indicando crescimento vegetativo da demanda —, a travessia Santos-Guarujá figurou na última posição do ranking estadual, com tendência negativa.

\begin{figure}[H]
	\centering
	\caption{Ranking de Tendência Secular das Travessias (2019-2025)}
	\includegraphics[width=0.9\linewidth]{assets/fig2_ranking.png}
	\fonte{Elaborado pelo autor.}
	\label{fig:ranking}
\end{figure}

Este dado aponta para uma estagnação estrutural no núcleo metropolitano da Baixada Santista. A retração do fluxo de automóveis sugere que a infraestrutura saturada pode estar expulsando usuários para rotas terrestres alternativas (Rodovia Cônego Domênico Rangoni), ainda que estas impliquem em maior distância percorrida.

\subsection{Determinantes do Fluxo: A Abordagem de Aprendizado de Máquina}

A aplicação do algoritmo \textit{Gradient Boosting} (XGBoost) permitiu ranquear as variáveis explicativas quanto à sua contribuição para a redução do erro de predição (Figura \ref{fig:xgboost}).

\begin{figure}[H]
	\centering
	\caption{Importância Relativa das Variáveis (XGBoost)}
	\includegraphics[width=0.8\linewidth]{assets/fig3_xgboost.png}
	\fonte{Elaborado pelo autor.}
	\label{fig:xgboost}
\end{figure}

A análise indicou que a variável temporal ($t$) e a sazonalidade (Mês) são os preditores dominantes. Notavelmente, a variável econômica (Importações/Valor FOB) apresentou baixa relevância explicativa, fornecendo a primeira evidência empírica do desacoplamento: para o algoritmo não-linear, as oscilações no valor movimentado pelo Porto contribuem marginalmente para explicar a variação no número de automóveis na balsa.

\subsection{Inferência Estrutural Bayesiana}

O modelo de regressão estrutural Bayesiana aprofundou essa análise, quantificando a incerteza e a direção dos vetores de força. O modelo final obteve um coeficiente de determinação ($R^2$) de 0,34 e um Erro Percentual Absoluto Médio (MAPE) de 9,26\%, demonstrando robustez no ajuste.

A Tabela \ref{tab:coeficientes} resume os parâmetros estimados.

\begin{table}[H]
	\centering
	\caption{Coeficientes Estruturais Estimados (Posterior Summary)}
	\label{tab:coeficientes}
	\begin{tabular}{lccc} 
		\toprule
		\textbf{Componente} & \textbf{Direção} & \textbf{HDI 95\%} & \textbf{Interpretação} \\ 
		\midrule
		Choque Pandemia & Negativo ($\downarrow$) & [-1,49; -0,34] & Retração Severa \\ 
		Tendência Secular & Negativa ($\downarrow$) & [-0,48; -0,06] & Declínio/Estagnação \\ 
		Impacto Econômico & Neutro ($\approx 0$) & [-0,01; 0,38] & Desacoplamento \\ 
		Sazonalidade (Dez) & Positivo ($\uparrow$) & [0,80; 1,55] & Pico Turístico Extremo \\
		\bottomrule
	\end{tabular}
	\fonte{Elaborado pelo autor com base no modelo PyMC.}
\end{table}

A análise visual dos coeficientes (Figura \ref{fig:bayes_fatores}) confirma três achados centrais:

\begin{figure}[H]
	\centering
	\caption{Fatores Estruturais: Tamanho do Efeito e Incerteza}
	\includegraphics[width=0.85\linewidth]{assets/fig4_fatores.png}
	\fonte{Elaborado pelo autor.}
	\label{fig:bayes_fatores}
\end{figure}

\begin{enumerate}
	\item \textbf{Desacoplamento Econômico:} O intervalo de credibilidade do impacto econômico cruza a linha do zero. Não há evidência estatística de que o crescimento das importações cause aumento no fluxo de automóveis.
	\item \textbf{Perfil Sazonal Assimétrico:} A modelagem por efeitos fixos mensais (Figura \ref{fig:sazonal}) revelou que a demanda não segue uma onda suave, mas sim picos abruptos em dezembro e janeiro, contrastando com vales profundos entre abril e junho.
\end{enumerate}

\begin{figure}[H]
	\centering
	\caption{Perfil Sazonal: Coeficientes de Efeito Mensal}
	\includegraphics[width=0.7\linewidth]{assets/fig5_sazonal.png}
	\fonte{Elaborado pelo autor.}
	\label{fig:sazonal}
\end{figure}

\subsection{Ajuste do Modelo e Projeção}

Por fim, a reconstrução da série temporal (Figura \ref{fig:projecao}) demonstra a aderência do modelo aos dados observados. A área sombreada representa o intervalo de incerteza (90\%).

\begin{figure}[H]
	\centering
	\caption{Decomposição Estrutural e Projeção (2019-2025)}
	\includegraphics[width=\linewidth]{assets/fig6_projecao.png}
	\fonte{Elaborado pelo autor.}
	\label{fig:projecao}
\end{figure}

Observa-se que, apesar da recuperação sazonal projetada para o final de 2025, a tendência média dos dados reais permanece pressionada, seguindo a leve declinação iniciada a partir da recuperação no fim de 2021, não tendo recuperado plenamente os patamares de volume pré-pandemia (2019), confirmando a hipótese de retração estrutural do sistema.
	
\section{Conclusão}

Este trabalho propôs uma investigação baseada em Ciência de Dados para elucidar a dinâmica de mobilidade na Travessia Santos-Guarujá, focando especificamente na demanda de automóveis. A aplicação de modelos de \textit{Gradient Boosting} e Inferência Bayesiana sobre dados de 2019 a 2025 permitiu refutar a hipótese de acoplamento econômico e revelar um cenário de saturação estrutural.

A principal contribuição empírica foi a demonstração estatística do desacoplamento entre a pujança do Porto de Santos e a mobilidade da travessia. O modelo Bayesiano indicou uma correlação fraca ($\beta_{econ} \approx 0$) entre as importações e o fluxo de veículos leves, sugerindo que a travessia opera sob uma dinâmica urbana saturada, insensível aos ciclos de bonança do comércio exterior.

Adicionalmente, a análise comparativa estadual revelou um dado alarmante: a travessia Santos-Guarujá apresentou uma tendência secular negativa ($\beta_{trend} < 0$) entre os principais sistemas litorâneos. Enquanto travessias periféricas registram crescimento de demanda, o sistema do estuário enfrenta estagnação e declínio no volume de automóveis transportadoss.

A modelagem da sazonalidade por efeitos fixos também evidenciou a extrema dependência dos meses de janeiro e dezembro, exigindo planejamentos operacionais que vão além da média anual.

\begin{thebibliography}{99}
	
	
	
	\bibitem{scielo_covid}
	ASSOCIAÇÃO NACIONAL DE TRANSPORTES PÚBLICOS (ANTP). \textbf{Impactos da Covid-19 no Transporte Público}. São Paulo: ANTP, 2020.
	
	\bibitem{bird_anyport}
	BIRD, J. H. \textbf{Seaports and seaport terminals}. London: Hutchinson, 1971.
	
	\bibitem{unisanta}
	PEDROSA, R. A.; NARDI, M. F.; OLIVEIRA, E. J. Os impactos da relação porto/cidade na cidade de Santos sob a perspectiva dos profissionais portuários. \textbf{Anais do Encontro Nacional de Pós-graduação}, Santos, v. 1, n. 1, p. 36-40, 2017.
	
	
	\bibitem{google_bsts}
	BRODERSEN, K. H. et al. Inferring causal impact using Bayesian structural time-series models. \textbf{Annals of Applied Statistics}, v. 9, n. 1, p. 247-274, 2015.
	
	\bibitem{xgboost_paper}
	CHEN, T.; GUESTRIN, C. XGBoost: A Scalable Tree Boosting System. In: SIGKDD CONFERENCE ON KNOWLEDGE DISCOVERY AND DATA MINING, 22., 2016, San Francisco. \textbf{Proceedings [...]} New York: ACM, 2016. p. 785–794.
	
	\bibitem{pymc}
	FONNESBECK, C. et al. \textbf{PyMC: Probabilistic Programming in Python}. Version 5.0. [S.l.]: PyMC Developers, 2024. Disponível em: https://www.pymc.io. Acesso em: 28 nov. 2025.
	
	\bibitem{seade_comex}
	FUNDAÇÃO SEADE. \textbf{Comércio Exterior dos Municípios Paulistas}. São Paulo: Seade, 2025. Disponível em: https://repositorio.seade.gov.br/dataset/comercio-exterior. Acesso em: 05 dez. 2025.
	
	\bibitem{gelman_bayes}
	GELMAN, A. et al. \textbf{Bayesian Data Analysis}. 3. ed. Boca Raton: CRC Press, 2013.
	
	\bibitem{tribuna_caos}
	FILAS sem fim e horas de espera: travessia de balsas entre Santos e Guarujá vive caos. \textbf{A Tribuna}, Santos, 04 dez. 2024. Disponível em: https://www.atribuna.com.br. Acesso em: 05 dez. 2025.
	
	\bibitem{hoylet}
	HOYLE, B. S. Global and local change on the port-city interface: the case of East African seaports. \textbf{GeoJournal}, v. 48, n. 2, p. 147-158, 1999.
	
	\bibitem{jornal_usp}
	MUDANÇAS no transporte coletivo aumentaram risco de contágio. \textbf{Jornal da USP}, São Paulo, 2020. Disponível em: https://jornal.usp.br. Acesso em: 10 jan. 2025.
	
	\bibitem{antagonismo}
	MODAL CONNECTION. \textbf{Túnel Santos-Guarujá: marco para a logística brasileira}. 2024. Disponível em: https://modalconnection.com.br/modais/tunel-santos-guaruja-marco-para-a-logistica-brasileira/. Acesso em: 05 dez. 2025.
	
	\bibitem{conflito_espaco}
	ORNELAS, R. S. \textbf{Relação porto/cidade: o caso de Santos}. 2009. Dissertação (Mestrado em Geografia Humana) - Universidade de São Paulo, São Paulo, 2009.
	
	\bibitem{pedrosa_santos}
	PEDROSA, R. A. Os impactos da relação porto/cidade na cidade de Santos sob a perspectiva dos profissionais portuários. \textbf{Revista Ceciliana}, Santos, v. 4, n. 1, p. 25-35, 2012.
	
	\bibitem{sklearn_gb}
	PEDREGOSA, F. et al. Scikit-learn: Machine Learning in Python. \textbf{Journal of Machine Learning Research}, v. 12, p. 2825-2830, 2011.
	
	\bibitem{semil_dados}
	SÃO PAULO (Estado). Secretaria de Meio Ambiente, Infraestrutura e Logística. \textbf{Volume Transportado nas Travessias Litorâneas}. São Paulo, 2025. Disponível em: https://dadosabertos.sp.gov.br/dataset/volume-trav-lit. Acesso em: 05 dez. 2025.
	
	\bibitem{scielo_mobilidade}
	SILVA, A. B. Impactos da Covid-19 na Mobilidade e na Acessibilidade. \textbf{Revista dos Transportes Públicos}, v. 42, p. 55-67, 2021.
	
	\bibitem{stitchfix}
	STITCH FIX. \textbf{Sorry ARIMA, but I'm Going Bayesian}. 2016. Disponível em: https://multithreaded.stitchfix.com/blog/2016/04/21/forget-arima/. Acesso em: 20 dez. 2025.
	
	\bibitem{stitchfix_arima}
	TAYLOR, S. J.; LETHAM, B. Forecasting at Scale. \textbf{The American Statistician}, v. 72, n. 1, p. 37-45, 2018.
	
\end{thebibliography}
\end{document}